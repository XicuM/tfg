Com ja s'ha vist al llarg de la memòria, encara queda treball per realitzar
tant a nivell de programari com de hardware. Així, el projecte del \emph{motor
drive} propi encara continua i es preveuen avenços en els propers mesos. En el
moment de l'escriptura de la memòria, l'equip està iterant el disseny les
plaques de potència, que es el pas previ al seu montatge i testeig físic. 

Quant l'implementació de l'algorisme de control i de la capa de seguretat es
poden destacar els següent treball futur:

\begin{itemize}
    \item 
        S'implementarà i es validarà la comunicació CAN amb els circuits de
        precàrrega i amb la PU del vehicle.
    \item
        Es programaran la subrutina d'aturada suau per evitar que l'algorisme
        de control rebi una parada amb un esglaó.
    \item
        Es treballarà en realitzar el sistema de detecció de l'angle i de la
        velocitat angular sobre FPGA. Actualment la detecció s'implementa en
        hardware mitjançant un circuit integrat especialitzat. El problema
        principal a solucionar és la baixa velocitat de de funcionament de
        l'integrat en comparació al de la FPGA.
    \item 
        Es valorarà implementar el \emph{hardware in the loop} en cas de voler
        millorar les prestacions de l'algorisme de control, ja que proporciona
        una forma de testejar l'algorisme independentment de la fase de
        desenvolupament del hardware.
    \item
        Es valorarà millorar l'eficiència energèntica de l'inversor realitzant
        una metodologia de testeig més intensiva.
\end{itemize}
