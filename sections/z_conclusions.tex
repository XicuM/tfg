En aquesta memòria s'ha explicat i analitzat el procés d'implementació d'un
control per \emph{Motor drive} en FPGA i de la capa de seguretat addicional,
posant especial èmfasi a les decisions presses en cada moment i com s'ha anat
desenvolupant el projecte en conseqüència. 

El projecte va iniciar-se amb bon peu en un principi. En la fase
intermitja es varen trobar algunes dificultats en la depuració del model en
Vitis Model Composer, en gran part pels temps de simulació elevats. Es va poder
completar aquest apartat; no obstant, els resultats obtinguts no acaben de ser
totalment rigurosos. En el cas d'haver-hi dedicat un temps addicional a
l'experimentació i l'anàlisi dels resultats en comptes de continuar amb la capa
de programari, es podria haver acabat de validar completament l'algorisme en
FPGA. No obstant, es decidí seguir amb el pla de treball original, ja que en el
moment de realitzar la revisió crítica es va subestimar la magnitud del treball
necessari per programar la capa d'arrencada, parada i seguretat, evidenciat en
aspectes com la dificultat de la implementació de la comunicació per bus CAN a
causa dels problemes de compatibilitat sorgits.

D'altra banda, l'ús de l'eina Vitis Model Composer ha resultat en bona mesura una
decisió acertada. Tanmateix, han sorgit certes dificultats que han empitjorat
el temps de depuració de l'algorisme:

\begin{itemize}

    \item \textbf{Paràmetres ocults.}
        Certa configuració que es troba oculta a primera vista, com és la
        configuració de la coma fixa o el \emph{sample rate} de cada bloc, amb
        la qual cosa es dificulta en certa mesura la depuració.

    \item \textbf{Bloqueig de Matlab.} 
        En realitzar les simulacions, moltes vegades Matlab es bloquejava i
        s'havia d'iniciar el programa des del principi, cosa que comportava una
        possible pèrdua de les dades.

    \item \textbf{Dificultat amb el control de versions.}
        El control de versions no es aplicable amb els fitxers \emph{.slx} que
        contenen la configuració dels models de Simulink, ja que no estan
        pensats per comparar-los a nivell de codi. En conseqüència, el
        control de versions amb eines com Git es coplica ja que no podem
        accedir i modificar les línies de codi que el configuren.

\end{itemize}

Finalment, volia destacar que personalment aquest projecte ha suposat un procés
d'aprenentatge constant, no només a nivell tècnic i d'assimilació de nous
conceptes sinó també a nivell d'autoconeixement, organització personal i gestió
de les expectatives. He après bastant entenent el per què no he pogut
aconseguir certes fites en el projecte que no aplicant el conceptes que ja
tenia, sense deixar de valorar la feina realitzada i la constància aplicada.
Puc dir amb seguretat que l'experiència obtinguda treballant aquest projecte
m'ajudarà en un futur a afrontar reptes més complexos i exigents.