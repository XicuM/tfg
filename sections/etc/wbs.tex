\newcounter{wpref}
\renewcommand{\arraystretch}{1.25}


% ----------------------------------------------------------------------
% WP1

\stepcounter{wpref}
\begin{center}
    \begin{tabular}{| p{8.5cm} | p{5.25cm} |}
        \hline
            \textbf{WP name:} 
                \newline \hspace*{0.3cm}
                \begin{minipage}[t]{8cm}
                    Estudi i investigació prèvia
                \end{minipage}
                \smallskip
            & 
            \textbf{WP ref:} 
                \newline \hspace*{0.3cm}
                \begin{minipage}[t]{8cm}
                    \arabic{wpref}
                \end{minipage}
            \\
        \hline
            \textbf{Short description:} 
                \newline \hspace*{0.3cm}
                \begin{minipage}[t]{8cm}
                    En aquest paquet de treball, es realitza un estudi de les
                    prestacions del disseny elaborat per l'equip amb l’objectiu
                    de trobar les seves deficiències i proposar millores o
                    redissenys que incrementin la qualitat del control.
                    Mentrestant, es realitza una investigació per a l’elecció
                    d’una placa FPGA en la qual implementar l’algorisme de
                    control i un resolver o encoder.
                \end{minipage}
                \smallskip
            &
            \textbf{Planned start date:} \newline \hspace*{0.3cm} 
                { 14/03/2022 } \newline
            \textbf{Planned end date:} \newline \hspace*{0.3cm} 
                { 23/04/2022 } \\
        \hline

            \textbf{Internal task T1:} 
                \newline \hspace*{0.3cm}
                \begin{minipage}[t]{8cm}
                    Estudi de les prestacions del disseny inicial
                \end{minipage}
                \smallskip

            \textbf{Internal task T2:} 
                \newline \hspace*{0.3cm}
                \begin{minipage}[t]{8cm}
                    Elecció d’una placa FPGA comercial
                \end{minipage}
                \smallskip

            \textbf{Internal task T3:}
                \newline \hspace*{0.3cm}
                \begin{minipage}[t]{8cm}
                    Elecció d’un resolver o encoder comercial
                \end{minipage}
                \smallskip
            & 
            \textbf{ Deliverables: }
                \begin{itemize}
                    \item { Model en Simulink (.xls) }
                    \item { Figures comentades generades per simulació }
                    \item { Fitxers VHDL (.hdl) }
                \end{itemize} \\
        \hline
    \end{tabular}
\end{center}


% ----------------------------------------------------------------------
% WP2 

\stepcounter{wpref}
\begin{center}
    \begin{tabular}{| p{8.5cm} | p{5.25cm} |}
        \hline
            \textbf{WP name:} 
                \newline \hspace*{0.3cm}
                \begin{minipage}[t]{8cm}
                    Impementació de l'algorisme
                \end{minipage}
                \smallskip
            & 
            \textbf{WP ref:}
                \newline \hspace*{0.3cm}
                \begin{minipage}[t]{8cm}
                    \arabic{wpref}
                \end{minipage}
            \\
        \hline
            \textbf{Short description:} 
                \newline \hspace*{0.3cm}
                \begin{minipage}[t]{8cm}
                    Aquest paquet de treball consisteix a realitzar el
                    redisseny (si escau) i la implementació, primer en Simulink
                    i després en codi VHDL, de l’algorisme de control.
                \end{minipage}
                \smallskip
            &
            \textbf{Planned start date:} \newline \hspace*{0.3cm} 
                { 28/02/2022 } \newline
            \textbf{Planned end date:} \newline \hspace*{0.3cm} 
                { 13/03/2022 } \\
        \hline

            \textbf{Internal task T1:} 
                \newline \hspace*{0.3cm}
                \begin{minipage}[t]{8cm}
                    Millora i implementació del SVPWM 
                \end{minipage}
                \smallskip

            \textbf{Internal task T2:} 
                \newline \hspace*{0.3cm}
                \begin{minipage}[t]{8cm}
                    Millora i implementació de les transformades de Clarke, de
                    Park i de Park inversa, i de les LUTs del sinus, del
                    cosinus, de l’arrel quadrada i de l’arrel quadrada
                    recíproca
                \end{minipage}
                \smallskip

            \textbf{Internal task T3:}
                \newline \hspace*{0.3cm}
                \begin{minipage}[t]{8cm}
                    Millora i implementació dels controls PI per al control del
                    corrent, Field Weakening i el limitador de torque
                \end{minipage}
                \smallskip

            \textbf{Internal task T4:} 
                \newline \hspace*{0.3cm}
                \begin{minipage}[t]{8cm}
                    Depuració del sistema complet
                \end{minipage}
                \smallskip

            \textbf{Internal task T5:}
                \newline \hspace*{0.3cm}
                \begin{minipage}[t]{8cm}
                    Millora i implementació del control del resolver
                \end{minipage}
                \smallskip
            &
            \textbf{ Deliverables: }
                \begin{itemize}
                    \item { Figures comentades generades per simulació }
                    \item { Taules comparatives }
                \end{itemize} \\
        \hline
    \end{tabular}
\end{center}


% ----------------------------------------------------------------------
% WP3 

\stepcounter{wpref}
\begin{center}
    \begin{tabular}{| p{8.5cm} | p{5.25cm} |}
        \hline
            \textbf{WP name:}
                \newline \hspace*{0.3cm}
                \begin{minipage}[t]{8cm}
                    Disseny i implementació en microprocessador 
                \end{minipage}
                \smallskip
            & 
            \textbf{WP ref:}
                \newline \hspace*{0.3cm}
                \begin{minipage}[t]{8cm}
                    \arabic{wpref}
                \end{minipage}
            \\
        \hline
            \textbf{Short description:} 
                \newline \hspace*{0.3cm}
                \begin{minipage}[t]{8cm}
                    Aquest paquet de treball consisteix a dissenyar,
                    implementar i validar la gestió de la comunicació i altres
                    interrupcions i events sobre el microcontrolador que
                    incorpora la placa FPGA.
                \end{minipage}
                \smallskip
            &
            \textbf{Planned start date:} \newline \hspace*{0.3cm} 
                { 22/04/2022 } \newline
            \textbf{Planned end date:} \newline \hspace*{0.3cm} 
                { 10/05/2022 } \\
        \hline

            \textbf{Internal task T1:} 
                \newline \hspace*{0.3cm}
                \begin{minipage}[t]{8cm}
                    Disseny de la màquina d’estats
                \end{minipage}
                \smallskip

            \textbf{Internal task T2:} 
                \newline \hspace*{0.3cm}
                \begin{minipage}[t]{8cm}
                    Implementació del protocol de comunicació CAN
                \end{minipage}
                \smallskip

            \textbf{Internal task T3:}
                \newline \hspace*{0.3cm}
                \begin{minipage}[t]{8cm}
                    Validació amb un CAN bus analyzer
                \end{minipage}
                \smallskip
            & 
            \textbf{ Deliverables: }
                \begin{itemize}
                    \item { Diagrama d’estats }
                    \item { Codi en C de la màquina d’estats }
                    \item { Captures de les trames de comunicació }
                \end{itemize} \\
        \hline
    \end{tabular}
\end{center}


% ----------------------------------------------------------------------
% WP4 

\stepcounter{wpref}
\begin{center}
    \begin{tabular}{| p{8.5cm} | p{5.25cm} |}
        \hline
            \textbf{WP name:} 
                \newline \hspace*{0.3cm}
                \begin{minipage}[t]{8cm}
                    Validació de l’algorisme de control en hardware
                \end{minipage}
                \smallskip
                \newline
            & 
            \textbf{WP ref:} 
                \newline \hspace*{0.3cm}
                \begin{minipage}[t]{8cm}
                    \arabic{wpref}
                \end{minipage}
            \\
        \hline
            \textbf{Short description:} 
                \newline \hspace*{0.3cm}
                \begin{minipage}[t]{8cm}
                    En aquest paquet de treball, es validaran les prestacions
                    de l’algorisme de control implementat. La validació es durà
                    a terme en tres fases: una primera en què se simula l’àrea
                    utilitzada i la temportizació mitjançant Vivado, el
                    software del fabricant Xilinx per al desenvolupament amb
                    FPGAs; una segona fase consistent en la validació sobre la
                    placa de potència i finalment una validació en bancada en
                    la que es posa a prova el control dels motors.
                \end{minipage}
                \smallskip
            &
            \textbf{Planned start date:} \newline \hspace*{0.3cm} 
                { 11/05/2022 } \newline
            \textbf{Planned end date:} \newline \hspace*{0.3cm} 
                { 08/06/2022 } \\
        \hline

            \textbf{Internal task T1:} 
                \newline \hspace*{0.3cm}
                \begin{minipage}[t]{8cm}
                    Simulacions en Xilinx Vivado
                \end{minipage}
                \smallskip

            \textbf{Internal task T2:} 
                \newline \hspace*{0.3cm}
                \begin{minipage}[t]{8cm}
                    Validació en placa de potència
                \end{minipage}
                \smallskip

            \textbf{Internal task T3:}
                \newline \hspace*{0.3cm}
                \begin{minipage}[t]{8cm}
                    Validació en bancada
                \end{minipage}
                \smallskip
            & 
            \textbf{ Deliverables: }
                \begin{itemize}
                    \item { Captures de la simulació }
                    \item { Informes autogenerats d'utilització i de temporització }
                    \item { Imatge d'arrencada en targeta SD (extensió .bin) }
                    \item { Captrues del monitors dels instruments de testeig }
                \end{itemize} \\
        \hline
    \end{tabular}
\end{center}


% ----------------------------------------------------------------------
% WP5 %

\stepcounter{wpref}
\begin{center}
    \begin{tabular}{| p{8.5cm} | p{5.25cm} |}
        \hline
            \textbf{WP name:} 
                \newline \hspace*{0.3cm}
                \begin{minipage}[t]{8cm}
                    Redacció de la documentació
                \end{minipage}
                \smallskip
            & 
            \textbf{WP ref:} 
                \newline \hspace*{0.3cm}
                \begin{minipage}[t]{8cm}
                    \arabic{wpref}
                \end{minipage}
            \\
        \hline
            \textbf{Short description:} 
                \newline \hspace*{0.3cm}
                \begin{minipage}[t]{8cm}
                    Aquest bloc de treball consisteix a recopilar el
                    coneixement generat, ordenar-lo i redactar la documentació
                    requerida.
                \end{minipage}
                \smallskip
            &
            \textbf{Planned start date:} \newline \hspace*{0.3cm} 
                { 28/02/2022 } \newline
            \textbf{Planned end date:} \newline \hspace*{0.3cm} 
                { 19/06/2022 } \\
        \hline

            \textbf{Internal task T1:} 
                \newline \hspace*{0.3cm}
                \begin{minipage}[t]{8cm}
                    Redacció de la proposta de projecte i pla de treball
                \end{minipage}
                \smallskip

            \textbf{Internal task T2:} 
                \newline \hspace*{0.3cm}
                \begin{minipage}[t]{8cm}
                    Redacció de la revisió crítica
                \end{minipage}
                \smallskip

            \textbf{Internal task T3:}
                \newline \hspace*{0.3cm}
                \begin{minipage}[t]{8cm}
                    Redacció de la memòria final
                \end{minipage}
                \smallskip
            & 
            \textbf{Deliverables: }
                \begin{itemize}
                    \item { Proposta de projecte i pla de treball }
                    \item { Revisió crítica }
                    \item { Memòria final }
                \end{itemize} \\
        \hline
    \end{tabular}
\end{center}