\subsection{ Filosofía de diseño }

Para el diseño, se han seguido una serie de principios a modo de guía en la
toma de decisiones:

\begin{itemize}

    \item \textbf{ Navaja de Ockam }. Las soluciones sencillas son preferibles
    a las complicadas, puesto a medida que se incorpora complejidad, los
    errores humanos y de concepto se ven multiplicados. Por esta razón, se han
    escogido soluciones sencillas en la medida de lo posible.

    \item \textbf{ Test Early, Test Often, Test Everything }. Es un principio
    heurístico que pone de manifiesto la importancia de diseñar e implementar
    tests de validación de código, la misma que cualquier otro componente de
    código: documentándolos e incluyéndolos en el control de versiones como si
    lo fueran. Con el objetivo de implementar los tests, es deseable que el
    código adquiera modularidad, la cual puede ser otorgada de manera intuituva
    mediante el paradigma de la programación orientada a objetos (OPP) que
    permiten lenguages como C++.

    \item \textbf{ Modularidad }.

\end{itemize}

En la implementación del control, cabe marcar claramente dos flujos de trabajo
diferentes, en función de si se realiza la programación del la lógica
progamable o el microprocesador.

Una vez entendido el control estudiado y simulada, ya es posible diseñar la
solución de control.

En el proceso de diseño no solo debemos considerar las especificaciones finales
del sistema a implementar, sino también los prototipos necesarios para validar
las ideas teóricas que lo soportan y reducir al máximo los posibles errores
que, entre otros, están causados por una fabricación defectuosa, falta de
habilidad o simplemente por el despiste humano.

El desarrollo del inversor se ve limitado por diferentes factores. Teniendo un
entendimiento de cuáles son esos factores, se puede realizar un diseño más
acertado. Estos son:

\begin{itemize}
    \item Los recursos ecómicos disponibles por el equipo. 
    \item Los conocimientos teóricos.
    \item Los conocimientos "prácticos".
\end{itemize}

Entre los factores que no limitan el desarrollo del inversor encontramos el
tiempo, puesto que a priori no hay una fecha en la cuál el inversor deba estar
listo y nos limite en cuanto a tiempo.

Bajo estas condiciones, los antiguos resposables del proyecto del inversor
siempre manifestaron su voluntad de estudiar e implementar lo último en el
estado del arte. Por eso se han elegido transistores MOSTET de Silicon Carbide
y el control por FPGA, no teniendo en cuenta la facilidad de la implementación.

Desde el principio entendí que sería un proyecto lo suficientemente interesante
como para merecer dedicarme a ello en este trabajo fin de grado, haciendo gala
de la peculiar mezcla de valentía, locura y excentricismo que escondo en algún
de mi psique, de la cual me enorgullezco a posteriori, una vez ya he terminado
el proyecto, porque mientras estoy trabajando en él, cuando las dificultades
crecen y se reproducen, las cosas no parecen tan claras.